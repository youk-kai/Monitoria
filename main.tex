\documentclass{article}
\usepackage[utf8]{inputenc}
\usepackage{amsmath}

\makeatletter
\renewcommand*\env@matrix[1][*\c@MaxMatrixCols c]{%
  \hskip -\arraycolsep
  \let\@ifnextchar\new@ifnextchar
  \array{#1}}
\makeatother

\title{Resolução de exercícios}

\date{}

\begin{document}

\maketitle{}

Qualquer dúvida, pergunte! Tirar dúvidas é a função primária do monitor.

\section{Lista 1}

4. A eliminação, seja qual for, se baseia em operações sobre matrizes. Isso significa que apenas troca de posição de linhas, escalamento de linha e soma de uma linha a outra escalada são operações válidas.

Para que uma matriz esteja em \textbf{forma escalonada} (escada), é necessário que cada linha possua mais zeros iniciais do que a anterior. Na \textbf{forma escalonada reduzida}, ainda se exige que cada pivot (a primeira entrada não nula de uma linha i.e. o primeiro número não nulo que se segue aos zeros iniciais) seja o único elemento não nulo de sua coluna. \vspace{2mm}
\par d) A matriz abaixo não está na forma escalonada porque o segundo ($a_{21}$) encontra-se na \textit{mesma} coluna que o primeiro ($a_{11}$).

\[ \begin{bmatrix} 
1 & 5 & 2 \\ 
2 & 0 & 1 \\ 
\end{bmatrix} \]

Na forma escalonada, temos

\[ \begin{bmatrix} 
1 & 5 & 2 \\ 
0 & 10 & 3 \\ 
\end{bmatrix} \]

Observe que nossa definição anterior não foi suficientemente forte para garantir que só exista uma forma escalonada da matriz, então poderemos obter várias respostas (convença-se que, na verdade, temos infinitas respostas). Essa observação também se aplica à forma escalonada reduzida, que será

\[ \begin{bmatrix} 
-2 & 0 & -1 \\ 
0 & 10 & 3 \\ 
\end{bmatrix} \] \vspace{2mm} \\  
8. Para que valores do parâmetro \textit{a} o sistema abaixo admite solução?

\[ \begin{cases} 6x + y = 7 \\ 3x + y = 4 \\ -6x - 2y = a
\end{cases}
\]

Primeiro, precisamos converter o sistema ao formato de matrizes:

\[ \begin{bmatrix} 6 & 1 \\ 3 & 1 \\ -6 & -2 
\end{bmatrix}
\begin{pmatrix} x \\ y \end{pmatrix} =
\begin{pmatrix} 7 \\ 4 \\ a \end{pmatrix}
\]

Executando a Eliminação de Gauss na matriz de coeficientes aumentada

\[ \begin{bmatrix}[cc|c]
6 & 1 & 7\\ 
3 & 1 & 4 \\
-6 & -2 & a
\end{bmatrix} \]

Obtemos

\[\begin{bmatrix}[cc|c]
6 & 1 & 7\\ 
0 & -1 & -1 \\
0 & 0 & 8 + a
\end{bmatrix} \]

O que implica $0x + 0y = 8 + a \Longleftrightarrow{} a = -8$. 
\vspace{5mm}
\\ 
10. O posto de uma matriz é o seu número de linhas não nulas após o escalonamento. Suponha uma matriz $A_{mxn}$ em forma escada. Temos que o sistema descrito por A é \vspace{2mm}

\par i. \textbf{Possível} quando posto(A) = posto(A$\mid$b).
\par \hspace{4mm} a) \textbf{Determinado} (uma solução) quando (n - posto(A)) $\geq$ 0. Convença-se que isso é equivalente a N(eq) $\geq$ N(vv)
\par \hspace{4mm} b) \textbf{Indeterminado} (infinitas soluções) quando (n - posto(A)) $<$ 0. Convença-se que isso é equivalente a N(eq) $<$ N(vv) \vspace{1mm}

\par ii. \textbf{Impossível} quando posto(A) $\neq$ posto(A$\mid$b). \vspace{4mm}

\par b) A forma escada da matriz aumentada é

\[ \begin{bmatrix}[ccc|c]
1 & 4 & 3 & $b_1$ \\
0 & -7 & -6 & $b_2$
\end{bmatrix} \]

É evidente que posto(A) = posto(A$\mid$b) independentemente do valor de $b_i$. Mais ainda, o número de colunas é maior que o posto i.e. o sistema é indeterminado ou, equivalentemente, admite infinitas soluções. 

\end{document}
